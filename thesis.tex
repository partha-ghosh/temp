% Created 2022-09-13 Tue 23:09
% Intended LaTeX compiler: lualatex
\documentclass[letterpaper, 12pt]{article}
\usepackage{graphicx}
\usepackage{longtable}
\usepackage{wrapfig}
\usepackage{rotating}
\usepackage[normalem]{ulem}
\usepackage{amsmath}
\usepackage{amssymb}
\usepackage{capt-of}
\usepackage{hyperref}
\usepackage[table, dvipsnames]{xcolor}
\usepackage{listings}
\usepackage{url}
\usepackage{soul}
\usepackage{amsthm}
\usepackage{amsfonts}
\usepackage{tcolorbox}
\usepackage{graphicx}
\usepackage{wrapfig}
\usepackage{listings}
\usepackage{framed}
\usepackage{float}
\usepackage{tikz}
\definecolor{navyblue}{HTML}{000080}
\hypersetup{colorlinks=true, citecolor=RoyalBlue, linkcolor=BrickRed, urlcolor=black} % by default sets linkcolor=BrickRed, citecolor=green, filecolor=cyan, menucolor=red, runcolor=cyan, urlcolor=magenta
\usepackage{booktabs}
\usepackage{longtable}
\usepackage{array}
\usepackage{multirow}
\usepackage{hhline}
\usepackage{caption}
\usepackage{diagbox}
\newcommand\textstyleSourceText[1]{\texttt{#1}}\makeatletter\newcommand\arraybslash{\let\\\@arraycr}\makeatother\renewcommand\arraystretch{1.3}\setlength\tabcolsep{0.01\textwidth}
\setlength{\LTpre}{20pt} \setlength{\LTpost}{-20pt}
\usepackage{enumitem}
\setlist[description]{leftmargin=0pt}
\setlist[itemize]{itemsep=0pt}
\setlist[enumerate]{noitemsep}
\theoremstyle{definition}
\newtheorem{theorem}{Theorem}[section]
\theoremstyle{definition}
\newtheorem{corollary}{Corollary}[theorem]
\theoremstyle{definition}
\newtheorem{lemma}[theorem]{Lemma}
\theoremstyle{definition}
\newtheorem{remark}{Remark}
\theoremstyle{definition}
\newtheorem{definition}[theorem]{Definition}
\lstset{language=,basicstyle=\small\ttfamily,breaklines=true,frame=single,numbers=left,morecomment=[l][\color{gray}]{//}}
\newenvironment{colbox}[1][green]{\colorlet{shadecolor}{#1!15}\begin{shaded}}{\end{shaded}}
\usepackage{/home/partha/Templates/org/defs-private}
\usepackage[left=3cm,right=3cm,top=3cm,bottom=4cm]{geometry}
\usepackage[backend=bibtex]{biblatex}
\addbibresource{../papers/thesis.bib}
\author{Partha Ghosh\\University of Tübingen}
\usepackage[parfill]{parskip}
\sethlcolor{Goldenrod}\renewcommand{\sout}{\hl}
\usepackage{fontspec}\setmainfont{Times New Roman}
\date{}
\title{Exploring Supervised and Self-supervised Learning in Autonomous Driving}
\hypersetup{
 pdfauthor={},
 pdftitle={Exploring Supervised and Self-supervised Learning in Autonomous Driving},
 pdfkeywords={},
 pdfsubject={},
 pdfcreator={Emacs 28.1 (Org mode 9.5.2)}, 
 pdflang={English}}
\begin{document}

\maketitle


\section*{Abstract}
\label{sec:org42b30e6}
\clearpage 

\section*{Acknowledgements}
\label{sec:orge316623}
I want to thank Bernhard Jaeger and Katrin Renz for supervising the project
and providing me with the necessary guidance and valuable support. I want to
thank Prof. Andreas Geiger for the helpful discussions. Finally, I want to
express my very profound gratitude to my parents for their continuous support
and encouragement throughout my entire life.


\clearpage 

\clearpage \tableofcontents \clearpage

\section{Introduction}
\label{sec:orgeefd681}
Autonomous vehicles are a promising technical solution to important problems in
transportation. Every year more than a million people die due to traffic accidents
[1] primarily caused by human error [2]. Automating driving has the potential to
drastically reduce these accidents. Additionally, self-driving cars could improve
the mobility of people who are not able to drive themselves. The use of supervised
machine learning has become the dominant approach to autonomous driving because
it can handle high-dimensional sensor data such as images well. To train machine
learning algorithms in an end-to-end fashion, meaning directly optimizing a neural
network to perform the full driving task, one needs demonstrations from an expert
driver. Industrial research often collects data from human expert drivers, but this
approach is expensive in terms of money and time.
Simulations [3, 4] are frequently used to perform research on autonomous driving
because new ideas can safely be tested in them. In simulations, an alternative to
human experts called privileged experts is available to perform the data collection
task. Privileged experts are computer programs that have direct access to the
simulator (e.g. knowing the positions of all cars), circumventing the challenging
perception task. These privileged experts can generate labeled data faster than
human experts and at basically no cost.


Deep policy learning makes promising progress to many visuomotor control tasks
ranging from robotic manipula- tion [20, 22, 25, 39] to autonomous driving [4,
47]. By learning to map visual observation directly to control action through a
deep neural network, it mitigates the manual design of controller, lowers the
system complexity, and improves generalization ability. However, the sample ef-
ficiency of the underlying algorithms such as reinforcement learning or
imitation learning remains low. It requires a significant amount of online
interactions or expert demon- strations in the training environment thus limits
its real- world applications.  Many recent works use unsupervised learning and
data augmentation to improve the sample efficiency by pre- training the neural
representations before policy learning.  However, the augmented data in
pretraining such as frames with random background videos [16, 17, 43] shifts
drasti-cally from the original data distribution, which degrades the overall
performance of the model. Also, it remains challenging to generalize the learned
weights to the real- world environment as it is hard to design augmentations
that reflect the real-world diversity. In this work, we explore pretraining the
neural representation on a massive amount of real-world data directly. Figure 1
shows some uncurated YouTube videos, which contain driving scenes all over the
world with diverse conditions such as different weathers, urban and rural
environments, and various traffic densities.  We show that exploiting such
real-world data in deep policy learning can substantially improve the
generalization ability of the pretrained models and benefit downstream tasks
across various domains.


Self-supervised learning (SSL) is an approach of machine learning to learn from
unlabeled data. In this learning approach, the learning happens in two
steps. First, the task is solved based on pseudo-labels which helps to
initialize the network weights. Second, the network is finetuned with ground
truth data.
Self-supervised learning (SSL) is a method of machine learning. It learns from unlabeled sample data. It can be regarded as an intermediate form between supervised and unsupervised learning. It is based on an artificial neural network.[1] The neural network learns in two steps. First, the task is solved based on pseudo-labels which help to initialize the network weights.[2][3] Second, the actual task is performed with supervised or unsupervised learning.[4][5][6] Self-supervised learning has produced promising results in recent years and has found practical application in audio processing and is being used by Facebook and others for speech recognition.[7] The primary appeal of SSL is that training can occur with data of lower quality, rather than improving ultimate outcomes. Self-supervised learning more closely imitates the way humans learn to classify objects.[8]


Learning representations from unlabeled data is a popular topic in visual
recognition. The learned representations are shown to be generalizable across
visual tasks ranging from image classification, semantic segmentation, to object
detection [5, 13, 18, 40, 42]. However, these methods are primarily built for
learning features for recognition tasks rather than control, where an agent acts
in an uncertain environment. Decision-making may not benefit from some visual
information, such as the abstraction of appearance and texture. For instance,
visual factors like lighting and weather may even become confounders in policy
learning, decreasing overall performance [46]. These factors are typically
unimportant to the driving job. The properties that are relevant to the output
action, on the other hand, must be understood.
On the contrary, it is crucial
to learn the features that matter to the output action. For example, at the
driving junction, the traffic light occupies only a few pixels in the visual
observation but has a significant impact on the driver’s actions.

In this work, we explore an existing self-supervised learning approach catered
to our autonomous driving framework. Moreover, we propose a new self-supervised
learning method that outperforms the prior work on the challenging NEAT
validation routes \cite{Chitta2021}. \sout{Also, we propose a data cleansing
technique for and stratified sampling in the dataloader for effective training
that improves the driving performance tremendously}.



In summary, the main contributions of this work are:
\begin{itemize}
\item \sout{We present a novel self-supervised learning approach for autonomous driving.}
\item +We propose a new data cleaning pipeline and stratified sampling in training.
\item We show how the \emph{inertia problem} can be addressed by data cleaning strategy.
\end{itemize}

We organize the structure of the thesis as follows. We first provide an overview
of the learning-based pipeline in SAP and its limitations in Section \ref{org074fe3a}. We
then introduce details of our methodology in Section \ref{orgcb46f01}, followed by the
description of the baseline methods and the effectiveness of our proposed model
compared to the baselines both quantitatively and qualitatively in Section
\ref{org79829d8}.


\section{Related Work \label{org074fe3a}}
\label{sec:orgd5ac5ec}
\subsection{Self-supervised Learning}
\label{sec:org5d98605}

\subsubsection{Emerging Properties in Self-Supervised Vision Transformers}
\label{sec:org3b2375f}

\subsubsection{Exploring Simple Siamese Representation Learning}
\label{sec:orgc26e2c2}

\subsubsection{Momentum Contrast for Unsupervised Visual Representation Learning}
\label{sec:org186555b}

\subsubsection{Self-training with Noisy Student improves ImageNet classification}
\label{sec:org65d1418}
\subsection{Self-supervised Learning in Autonomous Driving}
\label{sec:orgcc4866c}

\subsubsection{SelfD: Self-Learning Large-Scale Driving Policies From the Web}
\label{sec:org56d4187}

\subsubsection{Offline Visual Representation Learning for Embodied Navigation}
\label{sec:org548eccd}

\subsubsection{Action-Conditioned Contrastive Policy Pretraining}
\label{sec:orge44a748}

\section{Method \label{orgcb46f01}}
\label{sec:org84f9401}
Our goal is to facilitate training driving policies at scale.


\subsection{Problem Setting \label{org2b0df1b}}
\label{sec:org474d5b3}
We consider the task of point-to-point navigation in an urban setting where
the goal is to complete a given route while safely reacting to other dynamic
agents and following traffic rules. To achieve this, we consider the imitation
learning approach of learning policy as in self-driving it is easier for an
expert to demonstrate the desired behaviour rather than to specify a reward
function.

\textbf{Imitation Learning (IL):} The goal of IL is, for an agent to learn a policy
 \(\pi_{\v{\th}}\), that imitates the behavior of an expert \(\pi^{*}\). The agent
 learns to map an input to a navigational decision. In general, the decision
 may either be a low-level vehicle control action \cite{Codevilla2019}
 (e.g. steering, throttle and break) or a desired future trajectory relative
 to the ego-vehicle, i.e., a set of \(K\) waypoints
 \cite{Chen2019,Mueller2018} in the BEV space. In the latter case, future
 waypoints may be paired with a hand-specified or learned motion controller to
 produce the low-level action \cite{Chen2019,Mueller2018}. In this work, we
 focus on the later representation due to its interpretability and
 genralizability. To find the mapping, we consider the Behavior Cloning (BC)
 approach of IL. An expert policy is first rolled out to collect at each
 time-step, high-dimensional observations of the environment including front
 camera image, ego-vehicle position and orientation, high-level navigational
 command and high-level goal location provided as GPS coordinates etc. From
 these high-dimensional observations, we derive our dataset \(\DD = \set{(\bf
   x_{i}, \bf w_{i})}_{i=1}^{N}\in(\XX,\WW)\) of size \(N\), where the input
 \(\bf x\) is a subset of the observations \sout{(see ?)} and the corresponding
 expert trajectory \(\bf w\), defined by a set of 2D waypoints relative to the
 coordinate frame of the ego-vehicle in BEV space, i.e., \(\bf
   w=(x_{t},y_{t})_{t=1}^{T}\), are calculated from the ego-vehicle positions
 and orientations from the subsequent frames. Our goal is to find a decision
 making policy i.e. a waypoint prediction function \(\pi_{\v{\th}}:\XX\to\WW\)
 with learnable parameters \(\v{\th}\in\R^{d}\). In BC, the policy
 \(\pi_{\v{\th}}\) is learned by training a neural network in a supervised
 manner using the dataset, \(\DD\), with a loss function, \(\LL\) i.e.

\[\under{\tx{argmin }}{{\th}} \Ex_{(\bf x,\bf w)\sim \DD} [\LL(\bf
   w,\pi_{\th}(\bf x))].\]

We use the \(L_{1}\) distance between the predicted trajectory,
\(\pi_{\v{\th}}(\bf x)\), and the corresponding expert trajectory, \(\bf w\),
as the loss function. We assume access to an inverse dynamic model
\cite{10.2307/j.ctt183ph6v}, implemented as a PID controller \(\I\), which
performs the low-level control, i.e., steer, throttle and brake, provided the
future trajectory \(\bf w\). The action are dertermined as \(\bf a=\I(\bf w)\).

\textbf{Global Planner:} We follow the standard protocol of CARLA 0.9.10 and assume that
high-level goal locations G are pro- vided as GPS coordinates. Note that
these goal locations are sparse and can be hundreds of meters apart, as
opposed to the local waypoints predicted by the policy π .
\sout{For one of the self-supervised learning methods, we will slightly augment
the output space of this function in Section} \ref{org27e88c9}.

\subsection{Input and Output Parameterization}
\label{sec:orgfe3415d}
Input Representation: Following [45, 23], we convert the
LiDAR point cloud into a 2-bin histogram over a 2D BEV
grid with a fixed resolution. We consider the points within
32m in front of the ego-vehicle and 16m to each of the sides,
thereby encompassing a BEV grid of 32m × 32m. We di-
vide the grid into blocks of 0.125m × 0.125m which results
in a resolution of 256 × 256 pixels. For the histogram, we
discretize the height dimension into 2 bins representing the
points on/below and above the ground plane. This results in
a two-channel pseudo-image of size 256 × 256 pixels. For
the RGB input, we consider the front camera with a FOV
of 100◦ . We extract the front image at a resolution of 400
× 300 pixels which we crop to 256 × 256 to remove radial
distortion at the edges.
Output Representation: We predict the future trajectory
W of the ego-vehicle in BEV space, centered at the current
coordinate frame of the ego-vehicle. The trajectory is repre-
sented by a sequence of 2D waypoints, \{wt = (xt , yt )\}Tt=1 .
We use T = 4, which is the default number of waypoints
required by our inverse dynamics model.

\subsection{Waypoint Prediction Network \label{org27e88c9}}
\label{sec:org49c1e5b}
As shown in \sout{Fig. 2}, we pass the 512-dimensional feature vector through an MLP
(comprising 2 hidden layers with 256 and 128 units) to reduce its dimensionality
to 64 for computational efficiency before passing it to the auto-regressive
waypoint network implemented using GRUs \cite{Cho2014}. We initialize the hidden
state of the GRU with the 64-dimensional feature vector. The update gate of the
GRU controls the flow of information encoded in the hidden state to the output
and the next time-step. It also takes in the current position and the goal
location (Section \ref{org2b0df1b}) as input, which allows the network to focus
on the relevant context in the hidden state for predicting the next waypoint.
We provide the GPS coordinates of the goal location (transformed to the
ego-vehicle coordinate frame) as input to the GRU rather than the encoder since
it lies in the same BEV space as the predicted waypoints and correlates better
with them compared to representing the goal location in the perspective image
domain \cite{Chen2019}. Following \cite{Filos2020}, we use a single layer GRU
followed by a linear layer which takes in the hid- den state and predicts the
differential ego-vehicle waypoints \(\set{\d \bf w_{t}}_{t=1}^{T}\) for \(T=4\)
future time-steps in the ego-vehicle current coordinate frame. Therefore, the
predicted future waypoints are given by \(\set{\bf w_{t}=\bf w_{t-1} +\d \bf
w_{t}}_{t=1}^{T}\). The input to the first GRU unit is given as \((0,0)\) since
the BEV space is centered at the ego-vehicle’s position.

\textbf{Controller:} We use two PID controllers for lateral and lon- gitudinal control
to obtain steer, throttle and brake values from the predicted waypoints,
\(\set{\bf w_{t}}_{t=1}^{T}\). The longitudinal controller takes in the
magnitude of a weighted average of the vectors between waypoints of consecutive
time-steps whereas the lateral controller takes in their orientation. For the
PID controllers, we use the same configuration as in the author-provided
codebase of \cite{Chen2019}. Implementation de- tails can be found in the supplementary.
\subsection{Self-supervised Approaches}
\label{sec:orge71d24c}
In this section we will introduce our proposed method and also explore some
prior works in our self-driving framework.

\subsubsection{Learning with Pseudolabels}
\label{sec:orgfa68275}
\subsubsection{Deep Visual Odometry}
\label{sec:org55d7f3f}
In this section, we will introduce our proposed method for self-supervised
learning for autonomous driving. Our key idea is to generate pseudo-labels for
unlabeled driving scenes by exploiting a learning-based ego-motion estimation
method because of its desirable properties of robustness to image noise and
camera calibration independence. Moreover, in our expert driving dataset the
weather conditions and the time of the day changes from frame to frame. For this
reason learning-based method is the best way to estimate ego-motions than
classical methods. We use an end-to-end learning approach following
\cite{Wang2017, Zhai2019} to train the model to map directly from input image
pairs to an estimate of ego-motion (in our use case, estimate of relative
translation is sufficient). The model we used, is a two module Long-term
Recurrent Convolutional Neural Networks.

\textbf{Feature-encoding Module:} In order to learn the geometric relationships from
two adjacent images, we use the following CNN architechture, inspired by
FlowNetSimple architechture \cite{Fischer2015} ignoring the decoder part of it
and only focusing the on the convolutional encoder.

\begin{center}
\begin{longtable}{r|r|r|r|r|r}
\hline
\raggedleft Layer &
\raggedleft Kernel Size &
\raggedleft Padding &
\raggedleft Stride &
\raggedleft Max Pool &
\raggedleft\arraybslash Number of Channels\\\hline
\raggedleft Input &
\raggedleft - &
\raggedleft - &
\raggedleft - &
\raggedleft - &
\raggedleft\arraybslash 6\\
\raggedleft Conv1 &
\raggedleft $3 \x 3$ &
\raggedleft 1 &
\raggedleft 0 &
\raggedleft $2 \x 2$ &
\raggedleft\arraybslash 64\\
\raggedleft Conv2 &
\raggedleft $3 \x 3$ &
\raggedleft 1 &
\raggedleft 0 &
\raggedleft $2 \x 2$ &
\raggedleft\arraybslash 128\\
\raggedleft Conv3 &
\raggedleft $3 \x 3$ &
\raggedleft 1 &
\raggedleft 0 &
\raggedleft $4 \x 4$ &
\raggedleft\arraybslash 256\\
\raggedleft Conv4 &
\raggedleft $3 \x 3$ &
\raggedleft 1 &
\raggedleft 0 &
\raggedleft $4 \x 4$ &
\raggedleft\arraybslash 512\\
\raggedleft Conv5 &
\raggedleft $3 \x 3$ &
\raggedleft 1 &
\raggedleft 0 &
\raggedleft $4 \x 4$ &
\raggedleft\arraybslash 1024\\\hline
\caption{Configuration of the CNN}
\label{table1}
\end{longtable}
\end{center}

Contrary to DeepVO \cite{Wang2017} and PoseConvGRU \cite{Zhai2019}, which uses a
huge 10-layered CNN architechtures, we used a very simple and lightweight
5-layered CNN architechture as shown in Table \ref{table1}. Each layer is
followed by an application of ReLU nonlinear activation function. We keep the
kernel size to 3, padding size to 1 for all the layers. The channel dimension
doubles in each subsequent layers. We use maxpool of size 2 in the first 2
layers and use maxpool of size 4 for the rest of the layers. The reason behind
this is that, having small receptive field in the first 2 layers encourages the
network to learn about the fine-grained geometric details in the pair of images
which is essential for relative motion estimation. On the other hand, the large
receptive field in the later layers enforces the network to ignore the global
context and also helps in reducing the number of learnable parameters as well.


The feature-encoding module
encodes the short-term motion feature in an image pair, while the memory-
propagating module captures the long-term motion feature in the consecutive
image pairs.The visual memory is implemented with convolutional gated re-
current units, which allows propagating information over time. At each time
step, two consecutive RGB images are stacked together to form a 6 channels
tensor for module-1 to learn how to extract motion information and estimate
poses. The sequence of output maps is then passed through a stacked ConvGRU module to generate the relative transformation pose of each image pair. We also
augment the training data by randomly skipping frames to simulate the veloc-
ity variation which results in a better performance in turning and high-velocity
situations. Randomly horizontal flipping and temporal flipping of the sequences
is also performed. We evaluate the performance of our proposed approach on
the KITTI Visual Odometry benchmark. The experiments show a competitive
performance of the proposed method to the geometric method and encourage
further exploration of learning based methods for the purpose of estimating
camera ego-motion even though geometrical methods demonstrate promising
results.


Self-supervised learning aims at gathering more labels to unlabeled inputs so
that we can generate better representation of the input. In self-driving task
our output is the waypoints. Which can be estimated completely independent of
the driving task and we can generate accurate estimation of the waypoints.

There are two ways to generate better representations of the inputs
\begin{enumerate}
\item Contrastive learning
This kind of learning try to learn a representation that describe the image
as a whole. Does not find the representation that is actually important to
the downstream task.
\item by generating pseudolabels
\end{enumerate}


it also predicts
accurate ego-motion estimation, 
\subsubsection{SelfD}
\label{sec:orgdcd8b7f}
In this section we will discuss the pri
\subsubsection{Self-supervised Representation Learning}
\label{sec:org2eb2d5d}
\begin{enumerate}
\item Action Conditioned Policy Pretraining
\label{sec:org43c17fb}
\item DINO
\label{sec:org28e5a04}
\end{enumerate}
\subsection{Loss Function}
\label{sec:org3619825}
Following [8], we train the network using an L1 loss be-
tween the predicted waypoints and the ground truth way-
points (from the expert), registered to the current coordi-
nate frame. Let wtgt represent the ground truth waypoint for time-step t, then the loss function is given by:
L=
T
X
\begin{center}
\begin{tabular}{lllr}
 & wt − wtgt &  & 1\\
\end{tabular}
\end{center}
(5)
t=1
Note that the ground truth waypoints \{wtgt \} which are avail-
able only at training time are different from the sparse goal
locations G provided at both training and test time.
\subsection{Implementation Details}
\label{sec:orgf73ca7b}

\subsubsection{Input and Output Parameterization}
\label{sec:orgded5797}
Input Representation: Following [45, 23], we convert the
LiDAR point cloud into a 2-bin histogram over a 2D BEV
grid with a fixed resolution. We consider the points within
32m in front of the ego-vehicle and 16m to each of the sides,
thereby encompassing a BEV grid of 32m × 32m. We di-
vide the grid into blocks of 0.125m × 0.125m which results
in a resolution of 256 × 256 pixels. For the histogram, we
discretize the height dimension into 2 bins representing the
points on/below and above the ground plane. This results in
a two-channel pseudo-image of size 256 × 256 pixels. For
the RGB input, we consider the front camera with a FOV
of 100◦ . We extract the front image at a resolution of 400
× 300 pixels which we crop to 256 × 256 to remove radial
distortion at the edges.
Output Representation: We predict the future trajectory
W of the ego-vehicle in BEV space, centered at the current
coordinate frame of the ego-vehicle. The trajectory is repre-
sented by a sequence of 2D waypoints, \{wt = (xt , yt )\}Tt=1 .
We use T = 4, which is the default number of waypoints
required by our inverse dynamics model.

\subsubsection{Dataset}
\label{sec:orgda20bae}

\subsubsection{Data Preprocessing Pipeline \label{org28c4336}}
\label{sec:org1a2390d}


\subsubsection{Training and Inference}
\label{sec:orgbc7447e}

\section{Experimental Results \label{org79829d8}}
\label{sec:orge956a23}

In this section, we describe our experimental setup, com-
pare the driving performance of our approach against sev-
eral baselines, conduct an infraction analysis to study dif-
ferent failure cases, visualize the attention maps of Trans-
Fuser and present an ablation study to highlight the impor-
tance of different components of our model.

\subsection{Task}
\label{sec:org0ee386b}
We consider the task of navigation along a set of
predefined routes in a variety of areas, e.g. freeways, urban
areas, and residential districts. The routes are defined by a
sequence of sparse goal locations in GPS coordinates pro-
vided by a global planner and the corresponding discrete
navigational commands, e.g. follow lane, turn left/right,
change lane. Our approach uses only the sparse GPS lo-
cations to drive. Each route consists of several scenarios,
initialized at predefined positions, which test the ability of
the agent to handle different kinds of adversarial situations,
e.g. obstacle avoidance, unprotected turns at intersections,
vehicles running red lights, and pedestrians emerging from
occluded regions to cross the road at random locations. The
agent needs to complete the route within a specified time
limit while following traffic regulations and coping with
high densities of dynamic agents.




We consider the task of driving in an urban setting in the realistic 3D simulator
CARLA version 0.9.13. We use routes that cover both highway and residential
areas. Agents are supposed to follow routes which are provided as GPS waypoints
by an A\textsuperscript{*} navigational planner. The goal is to arrive at the target location in a given
amount of time while incurring as little infraction penalties as possible. Infractions
that are penalized are:
\begin{itemize}
\item Collision with a pedestrian
\item Collision with a vehicle
\item Collision with a static object
\item Running a red light
\item Running a stop sign
\item Driving on the wrong lane or sidewalk
\item Leaving the route specified by the navigational planner
\end{itemize}
Along the routes, there are dangerous scenarios specified which the agent needs to
resolve in order to safely arrive at his destination. There are currently 10 different
scenarios specified:
\begin{itemize}
\item Rubble on the road leading to a loss of control.
\item Leading vehicle suddenly performs an emergency brake.
\item A pedestrian hidden behind a static object suddenly runs across the street.
\item After performing a turn at an intersection, a cyclist suddenly drives across
the street.
\item A slow vehicle gets spawned in front of the agent.
\item A static object is blocking the street.
\item Crossing traffic is running a red light at an intersection
\item The agent must perform an unprotected left turn at an intersection with oncoming traffic
\item The agent must perform a right turn at an intersection with crossing traffic
\item Crossing an unsignalized intersection.
\end{itemize}


\subsection{Dataset}
\label{sec:orge49775b}
We use the CARLA [21] simulator for training
and testing, specifically CARLA 0.9.10 which consists of
8 publicly available towns. We use 7 towns for training
and hold out Town05 for evaluation. For generating train-
ing data, we roll out an expert policy designed to drive us-
ing privileged information from the simulation and store
data at 2FPS. Please refer to the supplementary material
for additional details. We select Town05 for evaluation
due to the large diversity in drivable regions compared to
other CARLA towns, e.g. multi-lane and single-lane roads,
highways and exits, bridges and underpasses. We consider
two evaluation settings: (1) Town05 Short: 10 short routes
of 100-500m comprising 3 intersections each, (2) Town05
Long: 10 long routes of 1000-2000m comprising 10 inter-
sections each. Each route consists of a high density of dy-
namic agents and adversarial scenarios which are spawned
at predefined positions along the route. Since we focus on
handling dynamic agents and adversarial scenarios, we de-
couple this aspect from generalization across weather con-
ditions and evaluate only on ClearNoon weather.
\subsection{Evaluation Metrics \label{orge3f31f2}}
\label{sec:orgd7dfc4f}
For the CARLA Autonomous Driving Leaderboard, the driving performance of an
agent is characterized by a set of chosen metrics that considers different
aspects of driving. While all routes have the same type of metrics, their
respective values are calculated separately. These metrics are as follows:

\subsubsection{Route Completion}
\label{sec:orgaa6f3ff}
Route Completion is the percentage of the route that is completed by an
agent. If an agent drives off-road, that percentage of the route will not be
considered towards the computation of the route completion score. Additionally,
the following events will interrupt the simulation, preventing the agent to
continue which will effectively reduce the route completion:

\begin{itemize}
\item Route deviation: If an agent deviates more than 30 meters from the assigned route.
\item Agent blocked: If an agent doesn’t take any actions for 180 simulation seconds.
\item Simulation timeout: If no client-server communication can be established in 60 seconds.
\item Route timeout: If the simulation of a route takes too long to finish.
\end{itemize}

\subsubsection{Infraction Score}
\label{sec:org7d3e26e}
Infraction Score is a penalty for infractions where the agent starts with
an ideal base score of 1.0. For every infraction, the score is multiplied by the
penalty coefficient of that infraction type. Ordered by their severity, the
penalty coefficients are as follows:
\begin{itemize}
\item Collision with a pedestrian: 0.50
\item Collision with a vehicle: 0.60
\item Collision with a static object: 0.65
\item Running a red light: 0.70
\item Running a stop sign: 0.80
\end{itemize}
Note that this means that subsequent infractions will have a lower impact due to the
multiplicative nature of the score.

\subsubsection{Driving Score}
\label{sec:org38f07b9}
Driving Score is the main metric for performance, serving as the product between
the route completion and the infractions penalty. It is calculated in the
following way: \[\tx{Driving Score}=\q1N\sum_{i=1}^{N}R_{i}P_{i}\] where \(N\) is the
number of routes, \(R_{i}\) the route completion percentage of the \(i\)-th route and
\(P_{i}\) the infraction penalty of \(i\)-th route. Note that, this is not the
same as multiplying the averaged route completion with the averaged infraction
score. The driving score is a normalized metric, meaning that the best possible
score is 100 and the worst score is 0. For the validation routes, we run them
with 3 different seeds and report the mean and standard deviation of the 3
scores averaged across the routes.

\subsection{Comparisons to the Baselines Methods \label{orgaf30037}}
\label{sec:orgcce4980}
\subsection{Ablation Studies}
\label{sec:orgc204e24}

\section{Conclusion}
\label{sec:orgae0d496}

We envision broad and easily deployable autonomous navigation systems. However,
access to resources and data limits the scope of the brittle autonomous systems
today.  Our SelfD approaches enables to significantly improve an initially
trained policy without incurring additional data col- lection or annotation
efforts, i.e., for a new platform, per- spective, use-case, or ambient
settings. Crucially, due to the proposed underlying model architecture, we do
not in- corporate camera parameters or configuration assumptions into the
monocular inference. As SelfD is self-improving, a future direction could be to
continue and learn from increas- ingly larger online datasets beyond what is
described in our study. While we emphasized efficient large-scale training in
our model development, how to best extend SelfD to more explicitly leverage
temporal demonstration data is still an open question which could be studied
further in the future.  Finally, beyond complex 3D navigation, it would be
inter- esting to explore the applicability of our proposed training framework
for learning various embodied tasks from unla- beled web data.

In our experiments we have only used front camera for navigational
decesion. Even though, it can successful in perfect route completion, it
sometimes results in infractions due to invisibilty for example, ego-vehicle
changes lane but the rear vehicle collide with it or the passenger starts
walking after the car already have gone past it.
\printbibliography
\end{document}